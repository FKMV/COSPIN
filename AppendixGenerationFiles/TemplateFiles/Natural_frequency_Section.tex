\chapter{Variation of 1st natural frequency with embedded pile length}\label{sec_4}

In Figure ~\ref{alpha_variation} and Figure ~\ref{mode_shape_variation} the relative natural frequency and the relative mode shape at MSL (relative to the values for an excessively long pile), respectively, are plotted as function of the embedded pile length. The figures provide information of the robustness present in the design and hence provides input in the selection of embedded pile length. With the selected embedded pile length, the design is far from the steep part of the curves, therefore indicating a robust design.
In Figure ~\ref{delta_alpha} the change in alpha per meter change in pile length is plotted as function of the embedded pile length. This has provided input to the optimum balance between wall thickness and pile length with respect to dynamic behaviour of the pile.

\begin{figure}[!htbp]
\includegraphics[width=0.9\textwidth]{AppendixGenerationFiles/ProjectLocation/NFA_1.png}
\caption{Variation in natural frequency relative to that of an excessively long pile against embedded pile length for WTG {\ID_location}. Characteristic soil properties adopted and with the effect of cyclic degradation considered. Soil springs are linearized at the representative FLS load level.}
\label{alpha_variation}\end{figure}

\begin{figure}[!htbp]
\includegraphics[width=0.9\textwidth]{AppendixGenerationFiles/ProjectLocation/NFA_2.png}
\caption{Variation in mode shape at MSL relative to that of an excessively long pile against embedded pile length for WTG {\ID_location}. Characteristic soil properties adopted and with the effect of cyclic degradation considered. Soil springs are linearized at the representative FLS load level.}
\label{mode_shape_variation}\end{figure}

\begin{figure}[!htbp]
\includegraphics[width=0.9\textwidth]{AppendixGenerationFiles/ProjectLocation/NFA_3.png}
\caption{Variation of $\Delta \alpha/\Delta$ L against embedded pile length for WTG {\ID_location}.}
\label{delta_alpha}\end{figure}